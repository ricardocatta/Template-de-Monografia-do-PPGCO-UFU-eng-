\chapter[Conclusão]{Conclusão}

O resultado do isomorfismo das equações de N-S para fluidos incompressíveis, eq. (\ref{n-s_potencial}), com a equação do potencial vetor eq. (\ref{potencial_vector_eq}), levanta uma dúvida que precisa ser investigada com mais atenção. Se a equação de N-S tem uma correspondência direta com a equação do potencial vetor, surge uma dúvida se N-S pode estar incompleta. Se a substituição do campo elétrico e magnético escritos como função do potencial vetor, fosse substituída na equação de Ampére-Maxwell, iria aparecer derivadas parciais de segunda ordem no tempo. Ou seja, não teria a correspondência direta com as equações de N-S. Podemos concluir que se a correspondência direta entre as equações dos fluidos e do eletromagnetismo existe, algo parece estar incompleto em N-S. Entendemos que o isomorfismo entre as duas teorias pode nos levar a um melhor entendimento sobre os mistérios por trás da turbulência.

A intuição que tivemos em investigar a existência de um tensor de tensões viscosas análogo ao tensor de Maxwell, se mostrou viável. Agora precisamos investigar em um trabalho futuro sobre a existência e unicidade da solução da equação (\ref{balanço_de_momento_fluido}). Um tratamento numérico também será proposto em um trabalho futuro.